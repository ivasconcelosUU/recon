\documentclass[a4paper,10pt]{article}
\usepackage[utf8]{inputenc}
\usepackage[T1]{fontenc}
\usepackage[ngerman]{babel}
\usepackage{algorithm}
\usepackage{amsmath,amssymb,amsthm}
\usepackage{array}
\usepackage{caption,float}
\usepackage{cmbright}
\usepackage{colortbl,longtable,tabu,tabularx}
\usepackage[inline]{enumitem}
\usepackage{fancyhdr}
\usepackage[sfdefault,light]{FiraSans}
\usepackage{lipsum}
\usepackage[top=1.5in, bottom=1.5in, left=1in, right=0.75in]{geometry}
\usepackage{graphicx}
\usepackage{hyperref}
\usepackage{multicol,multirow}
\usepackage{nicefrac}
\usepackage{pdfpages}
\usepackage{polynom}
\usepackage{setspace}
\usepackage{tikz}
\usepackage{titlesec}
\usepackage{wrapfig}
\usepackage{etex}
\usepackage{mathrsfs}
\usepackage[noend]{algpseudocode}
\usepackage{subfig}
\usetikzlibrary{automata,cd,arrows,decorations.pathreplacing,patterns}

% Typografische Einstellungen
\makeatletter
\def\mdseries@sf{m}
\def\bfseries@sf{m}
\setlist[enumerate]{parskip=0pt}
\setlist[enumerate]{itemsep}
\setlist[enumerate]{label=\normalfont\arabic*)}
\setlist[itemize]{label=$\triangleright$}
\setlist[description]{font=\normalfont}
\setlength{\parindent}{0pt}
\setlength{\parskip}{\baselineskip}
\setlength{\headsep}{5\baselineskip}
\onehalfspacing


% Kopfzeile
\pagestyle{fancy}
\fancyhf{}
\renewcommand{\headrulewidth}{0.1pt}
\fancyhead[L]{\textbf{Recon} \\ L. Plagwitz 2020}
\fancyhead[R]{\textbf{Linear Operator} \\ Seite \thepage/\pageref*{LastPage}}

% Eigene Farben
\definecolor{sectioning}{RGB}{9,96,142}
\definecolor{mainbox}{RGB}{225,227,227}
\definecolor{information}{RGB}{22,166,93}
\definecolor{warning}{RGB}{184,49,47}

% Stil von Sektionen


% Eigene Operatoren
\DeclareMathOperator{\Aut}{Aut}
\DeclareMathOperator{\argmin}{argmin}
\DeclareMathOperator{\car}{char}
\DeclareMathOperator{\Coind}{Coind}
\DeclareMathOperator{\dfk}{def}
\DeclareMathOperator{\End}{End}
\DeclareMathOperator{\Fix}{Fix}
\DeclareMathOperator{\Gal}{Gal}
\DeclareMathOperator{\ggT}{ggT}
\DeclareMathOperator{\GL}{GL}
\DeclareMathOperator{\Hom}{Hom}
\DeclareMathOperator{\id}{id}
\DeclareMathOperator{\Iso}{Iso}
\DeclareMathOperator{\img}{img}
\DeclareMathOperator{\inv}{inv}
\DeclareMathOperator{\Jac}{Jac}
\DeclareMathOperator{\kgV}{kgV}
\DeclareMathOperator{\Map}{Map}
\DeclareMathOperator{\Mat}{Mat}
\DeclareMathOperator{\ord}{ord}
\DeclareMathOperator{\proj}{proj}
\DeclareMathOperator{\rk}{rk}
\DeclareMathOperator{\sgn}{sgn}
\DeclareMathOperator{\SL}{SL}
\DeclareMathOperator{\Syl}{Syl}
\DeclareMathOperator{\Sym}{Sym}

\def\bb{\mathbb}
\def\bf{\mathbf}
\def\cal{\mathcal}
\def\emphb#1{{\mdseries\textcolor{sectioning}{#1}}}
\def\emphg#1{{\mdseries\textcolor{information}{#1}}}
\def\emphr#1{{\mdseries\textcolor{warning}{#1}}}
\def\Im{\operatorname{Im}}
\def\longmapsfrom{\mathrel{\reflectbox{\ensuremath{\longmapsto}}}}
\def\longtwoheadrightarrow{\longrightarrow\mathrel{\mspace{-22mu}}\rightarrow}
\def\mapsfrom{\mathrel{\reflectbox{\ensuremath{\mapsto}}}}
\def\mat{\mathbf}
\def\op{\operatorname}
\def\Re{\operatorname{Re}}
\def\rm{\mathrm}
\def\vec{\mathbf}
\def\BState{\State\hskip-\ALG@thistlm}

\newtheorem*{rem}{Erinnerung}
\newtheorem*{rmk}{Bemerkung}
\newtheorem{dfn}{Definition}
\newtheorem{example}{Beispiel}
\newtheorem{theorem}{Theorem}

% Tags der pdf-Datei
\hypersetup{
	pdftitle = {Masterarbeit - Thema Segmentierung},
	pdfsubject = {},
	pdfauthor = {Lucas Plagwitz}
}

% Titel, Autor, Datum
\title{Masterarbeit - Thema Segmentierung}
\author{Lucas Plagwitz}
\date{}

\begin{document}
	
\begin{center}
	\Large{Linear Operator} \\
	\normalsize	Lucas Plagwitz \\
	16. März 2020
\end{center}




\section{Motivation}

\section{Überblick der vorhandenen Operatoren}



\subsection{Differential Operator}
Um Ableitungen darstellen und numerisch berechnen zu können betrachtet man das Prinzip der Differenzenquotienten.
\begin{dfn}[Differenzenquotienten]
	Betrachte man $u \in \bb R^n$, so bezeichnet man
	\begin{enumerate}
		\item $D_h^+[u] = $
		\item
		\item
	\end{enumerate}
	als ...
\end{dfn}
Die direkte Implentatino dieser und ähnlicher Quotineten finden oft in PDEs statt. Für die Chan-Vese Segmentierung wurde dies direkt Implentiert. Siehe hierfür die image\_boundary.py.

Um dies zu diskretisieren stellt man üblicherweise Matrizen auf, welche wir im Folgenden definieren.
\begin{dfn}
	Der diskrete 1d zentrale Differenzenoperator ist gegeben durch 
	 \[
	 D_h^+ = \begin{bmatrix} 
	 1 & -1 &         &  0 \\
	   & 1  &  \ddots &    \\
	   &    & \ddots  & -1 \\
	 0 &    &         &  1 
	 \end{bmatrix}
		D_h^- = \begin{bmatrix} 
		1  &        &        & 0 \\
		-1 & 1      &        &   \\
		   & \ddots & \ddots &    \\
		0  &        & -1     & 1 
		\end{bmatrix}
	D_h = \begin{bmatrix} 
	0 & -1 &  & 0 \\
	1 & 0 &  \ddots & \\
	& \ddots & \ddots & -1 \\
	0 &      & 1 & 0 
	\end{bmatrix}
	\]
\end{dfn}
Alle drei Matrixmultiplikationen sind in der differential\_operator.py zu finden. Um hier zu unterscheiden und einheitlich mit der abstakten Klasse des linearen Operators zu bleiben, erhält der Operator ein extra Attribut 'mode' mit ['forward', 'backward', 'center']. Dies ist nicht zu verwechseln mit dem forward und backward des Operators. Hierzu später mehr.





\section*{Literatur}

	\label{LastPage}
\end{document}
